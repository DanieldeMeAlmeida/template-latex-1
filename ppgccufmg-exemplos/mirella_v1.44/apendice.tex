% Este comando encapsula o conjunto de ap�ndices. A sua fun��o � fazer com que
% a numera��o dos ap�ndices seja feita com letras mai�sculas (A, B, C, etc.) e
% a palavra "Ap�ndice" anteceda as entradas no Sum�rio.
\begin{appendices}

% Para cada ap�ndice, um \chapter
\chapter{Primeiro Ap�ndice}

De acordo com a ABNT:

\begin{quotation}
Ap�ndice (opcional): texto utilizado quando o autor pretende complementar sua argumenta��o. S�o identificados por letras mai�sculas e travess�o, seguido do t�tulo. Ex.: AP�NDICE A - Avalia��o de c�lulas totais aos quatro dias de evolu��o

Anexo (opcional): texto ou documento \textbf{n�o elaborado pelo autor} para comprovar ou ilustrar. S�o identificados por letras mai�sculas e travess�o, seguido do t�tulo. Ex.: ANEXO A - Representa��o gr�fica de contagem de c�lulas
\end{quotation}

Tais defini��es (e outras) podem ser encontradas na NBR 14724-2001 Informa��o e documenta��o - trabalhos acad�micos\footnote{http://www.firb.br/abntmonograf.htm}.



% Fim dos ap�ndices (usar apenas depois do �ltimo ap�ndice)
\end{appendices}


